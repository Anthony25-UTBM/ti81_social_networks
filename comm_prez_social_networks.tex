\documentclass{beamer}

\usepackage[utf8]{inputenc}
\usepackage[T1]{fontenc}
\usepackage[francais]{babel}
\usepackage{lmodern}
\usepackage{color}
\usepackage{fix-cm}
\usepackage{textpos}
\usepackage{eurosym}
\usepackage{multirow}
\usepackage{perpage}

% Redéfinis les marges des tableaux
\let\oldtabular=\tabular
\def\tabular{\small\oldtabular}
\renewcommand{\arraystretch}{1.5}

\usetheme{Warsaw}
\usecolortheme{orchid}

% Permet de réinitialiser les footnote à chaque frame. Nécéssite 2 compilations.
\MakePerPage{footnote}

\setlength{\TPHorizModule}{0.01\textwidth}
\setlength{\TPVertModule}{0.01\textheight}

\setbeamertemplate{navigation symbols}{}

\title[Réseaux sociaux et communication]{
    La communication via les réseaux sociaux}
\author[HOUDAYER \and RUHIER]{
	Benoit HOUDAYER\\
    \and
	Anthony RUHIER
}
\institute[TI81 - UTBM]{
    Communication (TI81) - Université Technologique de Belfort Montbéliard}
\date{Avril 2015}

\begin{document}

% Permet de masquer le comptage de slides
\bgroup
\makeatletter
\setbeamertemplate{headline}{}
\setbeamertemplate{footline}
{
  \leavevmode%
  \hbox{%
  \begin{beamercolorbox}[wd=.5\paperwidth,ht=2.25ex,dp=1ex,center]{title in head/foot}%
    \usebeamerfont{title in head/foot}\insertshorttitle
  \end{beamercolorbox}%
  \begin{beamercolorbox}[wd=.5\paperwidth,ht=2.25ex,dp=1ex,center]{date in head/foot}%
    \usebeamerfont{date in head/foot}\insertshortdate{}
%    \insertframenumber{} / \inserttotalframenumber\hspace*{2ex}
  \end{beamercolorbox}}%
}
\makeatother

\begin{frame}[noframenumbering]
 	\frametitle{}
	\titlepage
\end{frame}
\egroup

\begin{frame}[noframenumbering,plain]
    Qu'est ce qu'un réseau social
\end{frame}

\logo{\includegraphics[height=1cm]{images/logo-iut.eps}}


% Ajout du compteur de slides
\expandafter\def\expandafter\insertshorttitle\expandafter{%
      \insertshorttitle\hfill%
      \insertframenumber\,/\,\inserttotalframenumber}

\begin{frame}
	\begin{small}
	\frametitle{Sommaire}
	\tableofcontents
	\end{small}
\end{frame}

\AtBeginSection[]
{
	\begin{small}
	\begin{frame}
		\frametitle{Sommaire}
		\tableofcontents[currentsection]
	\end{frame}
	\end{small}
}


%%%% Includes des sections :
%%%%%%%%%%%%%%%%%%%%%%%%%%%%%%
%
    \section{Réseaux personnels}

\subsection{Présentation}
\begin{frame}
\frametitle{Présentation}
\begin{itemize}
    \itemsep1.5em
    \item Exemples :
        \begin{itemize}
            \itemsep0.5em
            \item Facebook
            \item Twitter
            \item Instagram
        \end{itemize}
    \item Quelques chiffres :
        \begin{itemize}
            \itemsep0.5em
            \item 75\% de la population mondiale sur les réseaux sociaux % vérifier chiffres
            \item 68\% des personnes en France
            \item 1h30 d'utilisation/jour
        \end{itemize}
\end{itemize}
\end{frame}

\subsection{Objectifs}
\begin{frame}
\frametitle{Objectifs}
\begin{itemize}
    \itemsep1.5em
    \item Pour la personne physique
        \begin{itemize}
            \itemsep0.5em
            \item Se divertir
            \item Communiquer
            \item Échanger, partager
        \end{itemize}
    \item Pour la personnes morale
        \begin{itemize}
            \itemsep0.5em
            \item Développement d'images de marque
            \item Communication avec les consommateurs
            \item Surveiller son e-reputation et celle de ses concurrents
            \item Une meilleure visibilité pour un moindre coût
        \end{itemize}
\end{itemize}
\end{frame}

% Ajouter 3ème section

    \section{Réseaux de contenu}

\subsection{Présentation}
\begin{frame}
\frametitle{Présentation}
\begin{itemize}
    \itemsep1em
    \item Exemples : Reddit, Slashdot, Digg
    \item Centré sur le contenu
    \item Regroupement par thème
    \item Personne anonyme, derrière un pseudonyme
    \item Liens inter-utilisateurs laissés en arrière plan
\end{itemize}
\end{frame}

\subsection{}
\begin{frame}
\frametitle{}
\begin{itemize}
    \itemsep1em
    \item Contenu trié par popularité
    \item Émergence de communautés et de centres d'intérêts
    \item Expression d'opinions
    \item Adjacence des communautés
\end{itemize}
\end{frame}

    \section{Réalisations/Projets}

\subsection{Mise en avant des réalisations}
\begin{frame}
\frametitle{Mise en avant des réalisations}
\begin{itemize}
    \itemsep1.5em
    \item Très peu d'informations sur l'équipe ou la personne
    \item Le jugement, la notation et les abonnements sur font sur les travaux
    \item Pousse au partage
    \item Se retrouver dans la masse de contenu
\end{itemize}
\end{frame}

\subsection{Thématiques larges}
\begin{frame}
\frametitle{Thématiques larges}
\begin{itemize}
    \itemsep1.5em
    \item Majoritairement utilisé dans les métiers de l'informatique. Exemples:
        \begin{itemize}
            \item Développement logiciel : Github
            \item Modélisation 3D
            \item Graphisme : Deviantart
        \end{itemize}
    \item Sites laissant l'utilisateur fournir le contenu et le critiquer
    \item Cuisine: proposition de recettes, commentaires et notes
\end{itemize}
\end{frame}


\end{document}
